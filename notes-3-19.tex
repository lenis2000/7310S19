\documentclass{../uva7310}

\begin{document}

\section{March 19}

Review:
\begin{enumerate}
    \item Measure and integral
    \item Radon Nikodym derivatives
    \item Stone-Weierstrass
\end{enumerate}

Let $(X,F,\mu)$ be a measure space. Then we also have $L^p$ and $L^\infty$ norms. The $L^p$ space is defined as
$L^p=\{F-\text{measureable functions}, f:X\rightarrow\RR: |f|^p\in L^1\}$. (To do : $1\leq p<q\leq\infty$ implies $L^q\subseteq L^p$)
Consider $\frac{1}{x}$ on $[0,1]$. 

Suppose $\mu(X)<\infty$, then $L^\infty$ is defined as $L^\infty=\{\text{ess}\sup |f|<\infty\}$ where
$\text{ess}\sup=\sup\{a:\exists\varepsilon>0,\;\mu(|f|^{-1}(a-\varepsilon,a+\varepsilon))>0\}$. This includes bounded a.e..

Let $V$ be a vector space, $||\cdot||:V\rightarrow\RR_{\geq0}$ be a function so that
\begin{itemize}
    \item $||x||=0\Rightarrow x=0$
    \item $||\lambda x||=|\lambda| ||x||$
    \item $||x+y||\leq ||x||+||y||$
\end{itemize}
Then $(V,||\cdot||)$ is a \textit{normed vector space}.

Define $||f||_p=\left(\int|f|^p\right)^{1/p}$ and $||f||_\infty = \text{ess}\sup |f|$. We need to show that these are
norms. The triangle inequality for $||f||_p$ is called the \textit{Minkowski inequality}.

A normed space is $\textit{Banach}$ if it is complete with respect to the metric $\rho(x,y)=||x-y||$ where $||\cdot||$
is the norm.

To do: $p>1$ implies $L^p$ is Banach.

Not the case that $L^p$ Banach for $p\leq 1$ under usual metric. For example with $p=1/2$, $(\sqrt{x}+\sqrt{y})^2=x+y+2\sqrt{xy}\not\geq x+y$.

Exercise: $||f-g||_p^p$ defines a metric.

If $(V,||.||)$ is a normed vector space, then if $V=\RR^n$, then all norms are equivalent (topologically). That is, there exists a $c>0$ such that for
all $x$, $c^{-1}||x||_2\leq ||x||_1\leq c||x||_2$. To prove, look at $\min_i/\max_i ||e_i||$ where $(e_i)$ is a basis of $\RR^n$.

When $X$ is the 2 point space under counting measure, $||(x,y)||_p= (|x|^p+|y|^p)^{1/p}\leq 1$ (see diagram). $p=2$, circle, $p=1$ diamond,
as $p$ grows, we get closer to a square with $p=\infty$ giving us a square (with max norm). When $p<1$, we get a crushed diamond.

Def: LEt $V$ and $W$ be vector spaces and let $T:V\rightarrow W$ be a linear operator. Then $||T||=\sup_{v\neq 0} \frac{||Tv||_W}{||v||_V}$.
If $||T||<\infty$, then $T$ is \textit{bounded}.

In finite dimensions, all operators are bounded.

Examples of operators: 

$V=\{(x_n)_{n\in\NN}: \left(\sum x_n^2\right)^{1/2}\}$, $T((x_n))=(nx_n)$.


$V=C^1([0,1])$, $W=C$, $T=\frac{d}{dx}:C^1\rightarrow C$ which is a dense domain. But $T$ is not bounded. Really just not always defined.

To get a truly unbounded operator, need axiom of choice (e.g. it is a myth). Take bases of $V$ and $W$, Banach spaces, $(e_\alpha)$,
$(f_\beta)$. Take subcollection $(e_i)\subseteq (e_\alpha)$ and map $e_i\mapsto ie_i$.

An operator is bounded iff it is continuous since $||T(v+h)-T(v)||=||T(h)||\leq ||T||\cdot ||h||$.

Given $V$, take $V^*$ to be the dual space (the space of bounded functionals on $V$) and give it the operator norm.

Def: A set $A\subseteq \RR^d$ is convex if for all $x,y\in A$, then for all $\alpha\in[0,1]$ $\alpha x+(1-\alpha)y\in A$.

Exercise: All convex sets are Borel measureable. (its true in $\RR^1$; slice $A$ into $A_{x_1}\subseteq\RR^{d-1}$ for fixed $x_1$)

Def: A function $f$ is convex if the epigraph, $\{(x,y):y\geq f(x)\}$ is convex. That is, if $f(\alpha x+(1-\alpha) y)\leq \alpha f(x)+(1-\alpha) f(y)$
for all $x,y$ and $\alpha\in[0,1]$. Alternatively, if for all $x,y,z$ with $x<y<z$ then $\frac{f(z)-f(y)}{z-y}\geq \frac{f(y)-f(x)}{y-x}$.

We now want to show that convex implies continuous. Suppose $f:\RR\rightarrow\RR$ and $x_2<x<y<x_1$. Then 
\[
    \frac{f(x_1)-f(y)}{x_1-y}\geq\frac{f(y)-f(x)}{y-x}\geq \frac{f(y)-f(x_2)}{y-x_2}
\]
and slopes increase in $x$ and are bounded. Ergo, it converges as $x\to y$ and $f$ is continuous with directional derivatives.
If $f:\RR^d\rightarrow\RR$ then $g(t)=f(x+tv)$ is convex in $t$ and is thus has directional derivatives.
Ergo, $D_v^\pm = \lim_{t\to 0\pm} \frac{f(x+tv)-f(x)}{t}$ and $f$ is continuous.

Since convex functions are continuous, they are Borel measureable.

Recall that if $f:X\rightarrow R$ we can push forward $\mu$ to $\RR$ with $f_*\mu$.

Jensen's inequality: If $f\in L^1(X,F,\mu)$ with $\mu(X)<\infty$ and $\varphi:\RR\rightarrow\RR$ is convex then
\[
    \varphi\left(\frac{1}{\mu(X)}\int fd\mu\right)\leq \frac{1}{\mu(X)}\int \varphi(f(x))\mu(dx).
\]
\begin{proof}
    Let $E=\frac{1}{\mu(X)}\int fd\mu$ and $\beta=D^-\varphi(E)=\lim_{\varepsilon\to 0-}\frac{\varphi(E+\varepsilon)-varphi(E)}{\varepsilon}\leq D^+\varphi(E)$.
    Then $\varphi(s)-\varphi(E)\geq \beta(s-E)$. Ergo, $\varphi(f(x))-\varphi(E)\geq\beta(f(x)-E)$. Thus, $\frac{1}{\mu(X)}\int d\mu(x)\geq 0$.
\end{proof}

Cor:
\begin{enumerate}
    \item Young inequality: If $\varphi(t)=e^t$ and $\mu(X)=1$ then $e^{\int f}\leq \int e^f$.
    \item AMGM inequality: Take $X=\{1,\ldots, n\}$ with corresponding measures $\alpha_1,\ldots,\alpha_n>0$ with $\sum a_i=1$.
    Then $y_i=e^{f(i)}\geq 0$ and $\prod y_i^{\alpha_i} \leq \sum \alpha_i y_i$.
\end{enumerate}

Overview

Normed spaces, Banach spaces, bdd operators

Convexity

$L^p$, $L^\infty$

Goals:

$L^q\subseteq L^p$ when $q>p>1$, Banach, $(L^p)^*$.

Jensen, (H\"older, Minkowski).

Suppose $f\in L^1\mapsto T(f)\in\RR$. Then $T(f)=\int f d\mu$ is an example of a functional. We can also have $T_A(f)=\int 1_A f$.
Can take $T_g(f)=\int f\bar{g} d\mu$ for $g\in L^\infty$.

We will see that $(L^1)^*=L^\infty$, but $(L^\infty)^*$.

\section{March 21}

Assume $\mu(X)$ finite.

Recall
\begin{itemize}
    \item Convex
    \item Jensen ($\mu(X)=1$) $\phi(\int fd\mu)\leq \int\phi(f(x))d\mu(x)$.
    \item Young
\end{itemize}

If $\phi:\RR\rightarrow\RR$ convex and $f$ is $L^1$, is $\phi\circ f\in L^1$? If $\phi\circ f\notin L^1$ then $\int \phi\circ f=+\infty$.

H\"older inequality

Def: $||f||_p=\left(\int|f|^p\right)^{1/p}$
Def: Conjugate, $p\to p'$ s.t. that $1/p+1/p'=1$ then $p'=p/(p-1)$.

Want to show $||f||_p$ is a norm in $L^p$.

H\"older inequality: If $f\in L^p$ and $g\in L^{p'}$, then $||fg||_1\leq ||f||_p||g||_{p'}$.

Cor (Cauchy-Schwarz): $p=2$, then $\int fg \leq \sqrt{\int f^2\int g^2}$ and $||x_1x_2+y_1y_2|\leq \sqrt{(x_1^2+x_2^2)(y_1^2+y_2)^2}$.

Cor: $p=1$, $p'=\infty$, then $\int |fg|\leq \int |f|\text{ess}\sup |g|$.

Proof of H\"older

Assume $p\in(1,\infty)$ and $f,g\geq 0$ with $||f||_p\neq 0$ and $||g||_{p'}\neq 0$. Define $F=f/||f||_p$ and $G=g/||g||_{p'}$.
Then we claim that $FG\leq \alpha F^p+(1-\alpha) G^{p'}$. Consider $F^{p\alpha}G^{p'(1-\alpha)}$ with $\alpha=1/p$. Then
AM-GM inequality yields $FG=F^{p/p}G^{p'(1-1/p)}\leq \frac{1}{p}F^p+(1-\frac{1}{p}) G^{p'}$.
Moreover $\int F^p=\int G^{p'}=1$ and so $\int FG\leq 1$. QED.

Minkowski inequality: $||f+g||_p\leq ||f||_p+||g||_p$.

Proof: We can assume that $p\in(1,\infty)$. Then $t\mapsto t^p$ is convex for such $p$. Therefore, $((f+g)/2)^p\leq f^p/2+g^p/2$
which implies $f+g\in L^p$. Furthermore, $(f+g)^p=f(f+g)^{p-1}+g(f+g)^{p-1}$ and $(f+g)^{(p-1)p'}=(f+g)^p\in L^1$. Thus we can
use H\"older to get that $\int |f(f+g)^{p-1}|\leq ||f||_p \cdot \left(\int|f+g|^p\right)^{1/p'}$. Similarly,
$\int|g(f+g)^{p-1}\leq ||g||_p\left(|f+g|^p\right)^{1/p'}$. Thus $\int |f+g|^p\leq ...$. Almost QED

Cor: $L^p$ is a normed vector space. (Above proof showed vector space followed by norm).

What about when $p<1$? $L^p$ is still a vector space, but $||\alpha||_p=|\alpha|||f||_p$ and $||f||_p=0\Rightarrow f=0$ a.e..
However, no triangle inequality. But $||f+g||_p\leq 2^{1/p-1} (||f||_p+||g||_p)$. This is a \textit{quasi-norm}.
We still have a metric on this space. This is left as an exercise.

Thm: $L^p$ is Banach for $1\leq p< \infty$.

Pf: Left to show that $L^p$ is complete with respect to norm. Suppose $(f_n)_{n\in\NN}$ is Cauchy in $L^p$. Ergo,
$(f_n)$ is Cauchy in $\mu$ by Chebyshev ($\varepsilon \mu(\{x:|g(x)|>\varepsilon\})^{1/p}\leq ||g|_p$ for all $\varepsilon>0$).
This implies that there exists a subsequence $(f_{n_j})$ that converges to $f$ a.e.. Want $f_n\to f$ in $L^p$.
We know that for $n,m>N$ we have $||f_n-f_m||_p<\varepsilon$. Fatou implies that when $m>N$,
$||f-f_m||_p \leq \liminf_{j:n_j>N} ||f_{n_j}-f_m||_p<\varepsilon$. Thus $f_m\to f$ in $L^p$. QED.

Inclusion: If $\mu(X)<\infty$ then $L^p\subseteq L^q$ for all $\infty\geq q>p\geq 1$.

Note: If $X=\RR$, $\mu$ is Lebesgue. Take $f=1/x\; 1_{x>1}$ then $f\in L^{1+\delta}$ and $f\notin L^1$.

Pf of inclusion: Take $r=q/p>1$, $r'$ such that $1/r+1/r'=1$.
Write $||f||_p=||f\cdot 1||_p=\left(\int |f|^p\cdot 1\right)^{1/p}\leq \left[  \left(\int|f|^{pq/p}\right)^{p/q}\mu(X)^{1/r'} \right]^{1/p}=k(X,p,q)||f||_q$
where $k$ is a constant dependent on $X$, $p$, and $q$. QED.

Riesz representation theorems:

Def: If $V$ is a Banach space, $V^*$ is the space of linear functionals $\phi:V\rightarrow\RR$ with $||\phi||_{V^*}=\sup_{x\neq 0}\frac{|\phi(x)|}{||x||}$.

What is $(L^p)^*$?

Hahn-Banach Thm: If $V$ is a Banach space and $U$ is a linear subspace over $\RR$ and $\phi:U\rightarrow\RR$ a linear functional
where $\phi(x)\leq p(x)$ for all $x\in U$ with $p$ convex, then there exists a functional $\psi:V\rightarrow\RR$ such that
$\psi_U=\phi$ and $\psi(x)\leq p(x)$ for all $x\in V$. 

Thm: $(L^p)^*=L^{p'}$ for $1\leq p<\infty$ and $(L^\infty)^*\neq L^1$.

IntuitionL Let $\lambda(f)$ be a functional with $f\in L^p$. Identify it with $\int fgd\mu$ where $g$ a fixed function in $L^{p'}$.

Pf: 
We start by showing $(L^\infty)^*$ is not $L^1$. Take $L^\infty([0,1])=V$ (with Lebesgue measure) and $U$ to be the set of functions continuous on $[0,1]$
except maybe at $x=1/2$. Define $\phi(f)=\lim_{x\to 1/2 +} f(x)-\lim_{x\to 1/2 -}f(x)$. By Hahn-Banach, this extends to $\psi$ on $V$.
Assume there exists an $h$ such that $\psi(f)=\int_0^1 h\cdot f d x$. We can take an open set $E\subseteq [0,1]$ with $f_n\to 1_E$ ($f_n\in C([0,1])$).
Then $\psi(f_n)=0$ but $\int h f_n\to \int h 1_E=0$ by DCT. Thus $h=0$ and $\psi=0$. But $\psi$ cannot be (see any function with a jump at 1/2).
Ergo, $(L^\infty)^*\neq L^1$.

We now show that $(L^p)^*=L^{p'}$. Let $\Lambda(f)$ be a functional. We want to find $g\in L^{p'}$ such that $\Lambda(f)=\int fg$.
Define a signed measure $\nu(A)=\Lambda (1_A_)$. Without loss of generality take $\mu(X)=1$ and $||\Lambda||=1$ (normalize).
Thus, $|\nu(A)|\leq ||1_A||_p=\mu(A)^{1/p}$. Moreover $\nu$ is $\sigma$-additive. So, $E=E_1\cup E_2\cup\ldots$ and
$1_E$ is the $L^p$ limit of $1_{E_1\cup \dots\cup E_n}$ and $\Lambda$ is continuous. Thus $\nu$ is indeed a signed measure
that is absolutely continuous with respect to $\mu$. Ergo, there exists a Radon-Nikodym derivative $h$ so that
$\nu(A)=\int_A h d\mu$. This further implies that for all $f\in L^p$, $\Lambda(f)=\int hf d\mu\leq ||f||_p$. We still need $h\in L^{p'}$.
We already know $h\in L^1$. The following lemma completes the proof.

Lem: $\int |hf|\leq c||f||_p$ for all $f\in L^p$ then $h\in L^{p'}$.

Pf: WLOG: assume $c=1$. Suppose $h_n\to h$ monotonically with all $h_n$'s bounded. Then $\int h_n^{p'}=\int h_n^{p'-1}h_n\leq ||h_n^{p'-1}||_p$ by
assumption. Then $(p'-1)p=p'$ so $\int h_n^{p'}\leq ||h_n||_{p'}^{p'/p}$. Thus $||h_n||_{p'}\leq ||h_n||_{p'^{1/p}}$ and
$||h_n||_{p'}\leq 1$. Ergo, $||h||_{p'}\leq 1$. QED.
\end{document}